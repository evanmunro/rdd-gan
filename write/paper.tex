\documentclass[12pt]{article}
\usepackage[margin=1in]{geometry}
\usepackage[utf8]{inputenc}
\usepackage{setspace} 
\usepackage{amsmath,amssymb,amsthm, bm, bbm} 
\usepackage{graphicx, xcolor}
\usepackage{natbib} 
\usepackage{enumerate} 
\usepackage{comment} 
\usepackage{hyperref} 
\hypersetup{colorlinks,citecolor=blue,urlcolor=blue,linkcolor=blue}
\theoremstyle{definition}
\newtheorem{theorem}{Theorem}[section]
\newtheorem{assumption}{Assumption}[section]
\newtheorem{definition}{Definition}[section]
\newtheorem{proposition}{Proposition}[section]
\newtheorem{corollary}{Corollary}[section]
\newtheorem{lemma}{Lemma}[section]
\newtheorem{property}{Property}[section]

\newcommand{\indep}{\perp \!\!\! \perp}


\bibliographystyle{aer}

\begin{document}

\title{A Systematic Comparison of Regression Discontinuity Design Estimators using WGANs  \thanks{
We thank Brad Ross for helpful comments and discussions. Code for the simulations in this paper is available at \texttt{https://github.com/evanmunro/rdd-gan}.}}

\author{ Guido Imbens
\and 
Evan Munro 
\thanks{Graduate School of Business, Stanford University }  } 

\date{ 
\today } 
\maketitle
\singlespacing 
\abstract{
TBD 
%  \\ 
\vspace{0in}\\
\noindent\textbf{Keywords: Regression Discontinuity Design}  \\
\noindent\textbf{JEL Codes: }   \\ 
} 


\newpage 
\onehalfspacing
\section{Introduction} 

\section{Simulation Design}

\subsection{Simulated Data} 

\section{RDD Estimators}  

\section{Simulation Results} 

\section{Conclusion} 

\newpage
\bibliography{sampleBib.bib}

\end{document} 