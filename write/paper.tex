\documentclass[12pt]{article}
\usepackage[margin=1in]{geometry}
\usepackage[utf8]{inputenc}
\usepackage{setspace} 
\usepackage{amsmath,amssymb,amsthm, bm, bbm} 
\usepackage{graphicx, xcolor}
\usepackage{natbib} 
\usepackage{enumerate} 
\usepackage{comment} 
\usepackage{algorithmic} 
\usepackage{hyperref} 
\usepackage{booktabs}
\hypersetup{colorlinks,citecolor=blue,urlcolor=blue,linkcolor=blue}
\theoremstyle{definition}
\newtheorem{theorem}{Theorem}[section]
\newtheorem{assumption}{Assumption}[section]
\newtheorem{definition}{Definition}[section]
\newtheorem{proposition}{Proposition}[section]
\newtheorem{corollary}{Corollary}[section]
\newtheorem{lemma}{Lemma}[section]
\newtheorem{property}{Property}[section]

\newcommand{\indep}{\perp \!\!\! \perp}


\bibliographystyle{aer}

\begin{document}

\title{Comparing Regression Discontinuity Design Estimators using WGANs  \thanks{
We thank Brad Ross for helpful comments and discussions. Code for the simulations in this paper is available at \texttt{https://github.com/evanmunro/rdd-gan}.}}

\author{ Guido Imbens
\and 
Evan Munro 
\thanks{Graduate School of Business, Stanford University }  } 

\date{ 
\today } 
\maketitle
\singlespacing 
\abstract{
TBD 
%  \\ 
\vspace{0in}\\
\noindent\textbf{Keywords: Regression Discontinuity Design}  \\
\noindent\textbf{JEL Codes: }   \\ 
} 


\newpage 
\onehalfspacing
\section{Introduction} 

\section{Simulation Design}

\subsection{Sharp Regression Discontinuity Design} 

Following \citet{imbens2008regression}, we introduce RD design in the context of the Rubin Causal Model. A researcher is interested in the causal effect of a binary intervention. $Y_i(0)$ and $Y_i(1)$ are potential outcomes for unit $i$. 

We observe
\[ Y_i = (1-W_i) Y_i(0) + W_i Y_i(1)\] 

where $W_i \in \{0, 1\}$ is the observed treatment. We also observe scalar covariate $X_i$. In RD design, the assignment to the treatment is determined by the value of $X_i$ being on one side of a fixed threshold. 
\[ W_i = \mathbbm 1 (X_i > c )\] 
The goal in sharp RDD estimation is to estimate 
\[ \tau = \mathbb E[ Y_i(1) - Y_i(0) | X_i = c] \] 

We design a simulation based on Wasserstein GANs to compare the performance of 9 different RDD estimators. 

\subsection{RDD Estimators} 

We evaluate the performance of the following seven estimators. 

\begin{enumerate} 
\item  \textbf{IK}: Local linear regression with \citet{imbens2012optimal} bandwidth and the triangular kernel. 
\item \textbf{CLL}: Local linear regression with \citet{calonico2014robust} bias correction and MSE-optimal bandwidth. 
\item \textbf{CLQ}: Local quadratic regression with \citet{calonico2014robust} bias correction and MSE-optimal bandwidth. 
\item \textbf{IW}: \citet{imbens2019optimized} finite sample minimax linear estimator. 
\item \textbf{AK}: \citet{armstrong2018optimal} finite sample minimax linear estimator. 
\item \textbf{GAM}: Data-driven method that fits a generalized additive model \citep{wood2017generalized} on either side of the cutoff. 
\item \textbf{IKHonest}: \citet{imbens2012optimal} bandwidth selection for the point estimate of the RDD parameter and \citet{imbens2019optimized} for confidence interval construction around that point estimate. 
\end{enumerate} 

\subsection{Simulating RDD Datasets Using GANs} 

For Sharp RDD, datasets are of the form $(\bm W, \bm Y, \bm X)$. $\bm W$ is an $n$-length vector of treatments, with $W_i = \bm X_i > c$ for $i \in [n]$. We can denote the empirical distribution of $\bm X_i$ as $\mathbb F^n_x$, the empirical distribution of $(Y_i, \bm X_i )| W_i = 1$ as $\mathbb F^n_{1}$ and the empirical distribution of $(Y_i, \bm X_i) | W_i = 0$ as $\mathbb F^n_{0}$.  

These datasets are used to estimate three Wasserstein GANs, which take as input random normal noise $\bm Z \sim \mathbb R^{d_g}$, where $d_g = 3$. The estimation procedure for WGANs is in Appendix \ref{sec:wgan}. 
\begin{itemize} 
\item $\hat g_x(z)$ is an unconditional WGAN estimated on samples from $\mathbb F^n_x$ that generates data $\tilde {\bm X}$
\item $\hat g_1(x, z)$ is a conditional WGAN estimated on samples from $ \mathbb F^n_{1}$  that generates a sample ${\tilde Y_1}$ conditional on some $x$ with $x>c$ 
\item $\hat g_0(x, z)$ is a conditional WGAN estimated on samples from $\mathbb F^n_{0}$  that generates ${\tilde Y_0}$ conditional on some $x$ with $x< c$ \end{itemize} 

Once the GANs are trained, we can calculate the ground truth for the GAN estimated distribution by sampling random noise $Z_q \sim \mathbb R^{d_g}$ and generating outcomes at the cutoff. We average the difference in outcomes at the cutoff for the treated and untreated samples over $Q$ samples, where $Q > 10^6$ is a very large number. 

\[ \tilde \tau = \frac 1Q\sum \limits_{q=1}^Q \hat g_1(c, Z_q) - \hat g_0(c, Z_q)\]  

As $Q \rightarrow \infty$, then $\tilde \tau = \mathbb E[\tilde Y_i(1) - \tilde Y_i(0) | \tilde X_i = c]$ where potential outcomes are defined by the estimated GANs $\hat g_1(x, z)$ and $\hat g_0(x, z)$.   


Then for $s \in [S]$, where $S = 10,000$ we run each estimation procedure on a finite sample of simulated data of size $[n]$ equal to that of the original dataset. For each replication of the simulation, we first simulate a WGAN dataset $(\tilde {\bm W^s}, \tilde {\bm Y^s}, \tilde {\bm X^s})$ of size $n$.  $\hat g_x(z)$ is used to generate covariates $\tilde X^s$. Treatments are $\tilde {\bm W^s} = \bm{\tilde X^s} > c$ . Then use $\hat g_1(x, z)$ for elements of $\tilde X^s$ that are above the cutoff and $\hat g_0(x, z)$ for elements of $\tilde X^s$ that are below the cutoff to generate corresponding outcomes $\bm {\tilde Y^s}$. Then for each estimator $k$ of the $K$ estimators, calculate point estimate $\hat \tau^k_s $ and 95\% confidence intervals $( b^k_s, a^k_s)$ on the data sample $(\tilde {\bm W^s}, \tilde {\bm Y^s}, \tilde {\bm X^s})$.  

We can then calculate a variety of useful metrics for each estimator over the $S$ simulations. 
\[ \mbox{MSE} = \frac 1S \sum \limits_{s=1}^S (\hat \tau^k_s - \tilde \tau)^2 \] 

\[ \mbox{Bias} =  \frac 1S \sum \limits_{s=1}^S (\hat \tau^k_s - \tilde \tau) \] 

\[ \mbox{Variance} = Var(\hat \tau^k_s) \] 

\[ \mbox{Coverage} = \frac 1S \sum \limits_{s=1}^S \mathbbm 1 ( b^k_s < \tilde \tau < a^k_s) \] 

\subsection{Datasets}

The simulation exercise used to calculate the coverage, mean-squared error, bias, and variance for each of estimator is repeated for three RDD datasets. 

\begin{enumerate} 
\item \textbf{LEE}: \citet{lee2010regression} uses data from pairs of U.S. congressional elections from 1946 to 1998. The running variable is the net vote threshold for the candidate's party in the previous congressional election. The cutoff is 0. The outcome is the net vote share for the candidate in the following election. The treatment effect is the effect of incumbency on electoral success, comparing elections for which the candidate's party narrowly won or narrowly lost in the previous year. There are 6,558 observations in the dataset. 
\item \textbf{JL-MATH}: The dataset is a record of third-grade student achievement in Chicago Public Schools from 1997 to 1999, from \citet{jacob2004remedial}. The running variable is the minimum of the math and reading score for an exam taken in the spring and the outcome is the math score for an exam taken after summer school. The cutoff is the passing threshold. There are 70,831 observations in the dataset.  The treatment effect is the average effect of summer school on math attainment, comparing students on either side of the threshold for mandatory summer school. There are 70,831 observations in the dataset. 
\item \textbf{M-MATH}: The dataset is a record of fifth-grade student achievement in an unspecified large school district in the northeast from 2001 to 2002, from  \citet{matsudaira2008mandatory}. The running variable is the minimum of the math and reading baseline score and the cutoff is the passing threshold. The outcome is the post summer school exam score. If a student fails either subject, they are meant to attend summer school, although the qualification is fuzzy for this dataset. The treatment effect is the average intent-to-treat effect of summer school on math attainment, comparing students on either side of the threshold for mandatory summer school. There are 68,798 observations in the dataset. 
\end{enumerate} 

The summary statistics for the real data and for a sample of generated data of size equal to the real data is in Table \ref{tab:summary} 

\begin{table} [ht] 
\centering 
\begin{tabular}{lll}
\toprule
  & Mean (s.d.) Real & Mean (s.d.) Generated\\
\midrule
LEE running variable & 0.13 (0.46) & 0.14 (0.45)\\
LEE outcome & 0.55 (0.24) & 0.56 (0.23)\\
JL-MATH running variable & 0.23 (0.98) & 0.17 (0.98)\\
JL-MATH outcome & -0.44 (0.98) & -0.44 (0.95)\\
M-MATH running variable & 10.32 (41.27) & 9.95 (40.11)\\
\addlinespace
M-MATH outcome & 0.00 (1.00) & -0.08 (0.97)\\
\bottomrule
\end{tabular}
\label{tab:summary} 
\caption{Mean and standard deviation of real and generated data} 
\end{table} 

The estimated treatment effect for a real data and for a sample of generated data of size equal to the real data is in Table \ref{tab:tesample}. The characteristics of the generated data is very similar to that of the real data. 
\begin{table} [ht] 
\centering  
\begin{tabular}{lll}
\toprule
  & $\hat \tau$ (s.e.) Real & $\hat \tau$ (s.e.) Generated\\
\midrule
LEE  & 0.07992 (0.00835) & 0.08311 (0.00618)\\
 JL-MATH & -0.18381 (0.01666) & -0.19529 (0.01621)\\
M-MATH  & -0.08031 (0.01135) & -0.06149 (0.01157)\\
\bottomrule
\end{tabular}
\label{tab:tesample} 
\caption{Estimated treatment effects for real and generated data. The estimate is calculated using local-linear regression with the \citet{imbens2012optimal} bandwidth.} 
\end{table} 

\section{Simulation Results} 

The RMSE, bias, standard deviation, coverage and interval widths for each dataset for 1000 samples from the WGAN-generated dataset are recorded in Table \ref{tab:leemc}-\ref{tab:matsmc}. The bound on the second derivative for the IW and AK estimators is calculated in each step by estimating a global quadratic regression above and below the cutoff and evaluating the maximum second derivative of that polynomial in the range of the data. For LEE and JL-MATH, the GAM-based estimator has the lowest RMSE, and local-linear regression with the IK-bandwidth is a close second. For M-MATH the local linear regression has the lowest RMSE. The bias-corrected local linear and local quadratic estimators have a higher RMSE, and for the JL-MATH dataset, the local quadratic estimator substantially undercovers. The finite-sample minimax linear estimators, IW and AK, also have a higher For every dataset, either the bias-corrected local quadratic estimator or the AK estimator have the highest RMSE. 
\begin{table} [ht]
\centering 
\begin{tabular}{lrrrrrr}
\toprule
  & est & rmse & bias & sd & covg & width\\
\midrule
rddIK & 0.0829 & 0.0069 & -0.0025 & 0.0064 & 0.9439 & 0.0257\\
rddCLL & 0.0857 & 0.0103 & 0.0003 & 0.0103 & 0.9481 & 0.0394\\
rddCLQ & 0.0856 & 0.0126 & 0.0002 & 0.0126 & 0.9439 & 0.0470\\
rddIW & 0.0850 & 0.0100 & -0.0004 & 0.0100 & 0.9724 & 0.0453\\
rddAK & 0.0853 & 0.0112 & -0.0001 & 0.0112 & 0.9707 & 0.0487\\
\addlinespace
rddGAM & 0.0845 & 0.0068 & -0.0009 & 0.0068 & 0.9539 & 0.0272\\
rddIKHonest & 0.0829 & 0.0069 & -0.0025 & 0.0064 & 0.9941 & 0.0626\\
\bottomrule
\end{tabular}
\caption{Simulation Results for \textbf{LEE} Generated Data} 
\label{tab:leemc} 
\end{table} 

\begin{table} [ht]
\centering 
\begin{tabular}{lrrrrrr}
\toprule
  & est & rmse & bias & sd & covg & width\\
\midrule
rddIK & -0.2018 & 0.0173 & 0.0068 & 0.0159 & 0.9419 & 0.0656\\
rddCLL & -0.2068 & 0.0188 & 0.0019 & 0.0187 & 0.9556 & 0.0728\\
rddCLQ & -0.1904 & 0.0307 & 0.0182 & 0.0247 & 0.8846 & 0.0922\\
rddIW & -0.1925 & 0.0262 & 0.0162 & 0.0206 & 0.9240 & 0.0922\\
rddAK & -0.1884 & 0.0310 & 0.0202 & 0.0236 & 0.9161 & 0.1028\\
\addlinespace
rddGAM & -0.2159 & 0.0164 & -0.0073 & 0.0147 & 0.9634 & 0.0682\\
rddIKHonest & -0.2018 & 0.0173 & 0.0068 & 0.0159 & 1.0000 & 0.1477\\
\bottomrule
\end{tabular}
\caption{Simulation Results for \textbf{JL-MATH} Generated Data} 
\label{tab:jlmc} 
\end{table} 

\begin{table} [ht]
\centering
\begin{tabular}{lrrrrrr}
\toprule
  & est & rmse & bias & sd & covg & width\\
\midrule
rddIK & -0.0339 & 0.0109 & -0.0015 & 0.0108 & 0.9598 & 0.0454\\
rddCLL & -0.0296 & 0.0149 & 0.0027 & 0.0147 & 0.9547 & 0.0573\\
rddCLQ & -0.0259 & 0.0169 & 0.0064 & 0.0156 & 0.9631 & 0.0648\\
rddIW & -0.0289 & 0.0140 & 0.0034 & 0.0136 & 0.9782 & 0.0630\\
rddAK & -0.0260 & 0.0170 & 0.0064 & 0.0157 & 0.9740 & 0.0734\\
\addlinespace
rddGAM & -0.0407 & 0.0141 & -0.0084 & 0.0113 & 0.9623 & 0.0586\\
rddIKHonest & -0.0339 & 0.0109 & -0.0015 & 0.0108 & 1.0000 & 0.2739\\
\bottomrule
\end{tabular}
\caption{Simulation Results for \textbf{MATS-MATH} Generated Data} 
\label{tab:matsmc} 
\end{table} 

\subsection{Discussion} 


Both local linear regression and the two minimax linear estimators can be written as: 

\[ \hat \tau = \sum \limits_{i=1}^n \hat \gamma_i Y_i \] 

The IW estimator and the AK estimator select weights that minimize the worst-case RMSE over all possible smooth conditional mean functions in a restricted class. For the IW estimator, the conditional mean function has a bounded second derivative everywhere. For AK, the conditional mean function has a bounded second derivative at the cutoff. We see that these estimators have a much higher RMSE than the data-driven estimates. It is worth inspecting the weights of these estimators to explain why the optimal worst-case performance does not translate to better performance in practice. 

In Figure \ref{fig:weights} we plot the weights of the IK estimator and the IW estimator for the LEE dataset. The IK estimator chooses a fairly wide bandwidth that leads to some bias but a lower standard deviation in simulations. In contrast, the IW wager, which must be robust to conditional mean functions that are very curved near the cutoff, selects weights that are much more narrow around the cutoff. This leads to a higher standard deviation in simulations, even for datasets like LEE where these estimators have very low bias. 

\begin{figure}[ht] 
\centering
\includegraphics[width = 0.5\textwidth]{lee_weights.pdf} 
\caption{Weights for each RDD estimator near the cutoff for LEE data} 
\label{fig:weights} 
\end{figure} 


\section{Conclusion} 

\newpage
\bibliography{sample.bib}


\appendix

\section{Wasserstein GANs} 


\label{sec:wgan} 


\end{document} 